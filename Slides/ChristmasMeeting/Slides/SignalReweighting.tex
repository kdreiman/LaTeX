\begin{frame}
	
		\frametitle{Signal Sample Reweighting}
		\vspace{30pt}
		
		\begin{itemize}
		
		\item A `flat' signal sample is produced by removing the Breit-Wigner term from the cross section calculation in Pythia and fitting the resulting mass spectrum to smooth out bumps. 
		\item These terms must be replaced using the desired pole mass in order to obtain signal peaks.
		\end{itemize}
		
\begin{equation*}
	\Gamma = \frac{1}{\sin^{2}\Theta_{W} \times ( \alpha_{EM}(m_{Z})^{-1} + 1.45 log( \frac{m_{Z}}{M}) )} \times \frac{3 + ( 1 + \frac{rtW}{2}) \times ( 1 - rtW )^{2} }{4}
\end{equation*}



\begin{equation*}
	W_{BW} = \frac{1}{(m_{\ell\nu}^{2} - M^{2})^{2} + (m_{\ell\nu}^{2} \times \Gamma)^{2}}
\end{equation*}

Where M = desired polemass, $m_{\ell\nu}$ = invariant mass and $rtW = \Big(\frac{m_{t}}{M}\Big)^{2}$
		
		
%	\begin{itemize}
%		\item 
%		
%	
%\end{itemize}			
		
		
	
	\end{frame}
	
	
		\begin{frame}
	
		\frametitle{Signal Sample Reweighting}
		
		
	
		
		\li{$m_{\ell\nu}$-dependent weights are calculated and applied to the signal sample. }
		
		\begin{equation*}
  w = 
  \begin{cases}
    10^{12} \times 102.77 \exp\Big( -11.5 \frac{m_{\ell\nu}}{\sqrt{s}}\Big) \times W_{BW} & \text{if}\,\, m_{\ell\nu} < 299\, \text{GeV},\\
    10^{12} \times \exp\Big( -16.1 \frac{m_{\ell\nu}}{\sqrt{s}}\Big) \times \Big(\frac{m_{\ell\nu}}{\sqrt{s}}\Big)^{1.2} \times W_{BW} & \text{if}\,\, m_{\ell\nu} \geqslant 299\, \text{GeV}, m_{\ell\nu} < 3003\, \text{GeV},\\
    10^{12} \times 1.8675 \exp\Big( -31.7 \frac{m_{\ell\nu}}{\sqrt{s}}\Big) \times \Big(\frac{m_{\ell\nu}}{\sqrt{s}}\Big)^{4.6} \times W_{BW} & \text{if}\,\, m_{\ell\nu} \geqslant 3003\, \text{GeV}.
  \end{cases}
\end{equation*}

		
	
	\end{frame}