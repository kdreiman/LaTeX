\begin{frame}
	\frametitle{Limit Setting Procedure}
	\vspace{10pt}
	\begin{enumerate}
	\item{For each mass point, p values are calculated for a range of signal scale factor ($\mu$) guesses.}
	\li{These guesses were chosen by taking the asymptotic limit setting result for the workspace in question at each mass.}
	\lii{See next slide.}
	\li{When running the jobs a number of pseudo-experiments is chosen, split into n jobs.}
	
	\item{For each mass, a fit is performed for SF vs. p value.}
		\li{Using a TSpline3 fit.}
		\li{The x-position where the fit crosses p-value=0.05 is the 95\% CL limit value.}

	\end{enumerate}
\end{frame}



	\begin{frame}
	\frametitle{Method for Setting $\mu$ Range}
	
	\vspace{10pt}
	\li{A range of 31 \textmu\, values is chosen:}
	
	
	
	\vspace{5pt}
	\cleft{.2}
	\cright{.6}
	{\scriptsize{
	
	
	for m $\leqslant$ 150 : sfLo = $5e^{-5}$, sfHi = $5e^{-3}$\\
	\vspace{5pt}
	for m $\leqslant$ 5000 :
	
	\hspace{20pt}    $sfLo = Exp_{asymptotic} - \bigg( 3 \times \frac{Exp_{asymptotic}}{5} \bigg)$ \\
	\hspace{20pt}    $sfHi = Exp_{asymptotic} + \bigg( 31 \times \frac{Exp_{asymptotic}}{5} \bigg)$\\
  	\vspace{5pt}
	for m $>$ 5000: 
	
	
	\hspace{20pt}    $sfLo = Exp_{asymptotic} - \bigg( 2 \times \frac{Exp_{asymptotic}}{4} \bigg)$ \\
    \hspace{20pt}   $sfHi = Exp_{asymptotic} + \bigg( 34 \times \frac{Exp_{asymptotic}}{4} \bigg)$\\
   
	$sfStep = \frac{sfHi-sfLo}{30}$
	
	}}
	
	\cright{.2}
	\cend

	\vspace{5pt}
	\li{The expected value is either extracted directly from the asymptotics curve or by extrapolating a fit for asymptotics up to 2.5 TeV.}
	
\end{frame}



\begin{frame}
	\frametitle{Method For Setting $\mu$ Range}
	
	\cleft{.5}
		\begin{center}
		$\zprime \rightarrow ee$
		\includegraphics[width=\linewidth]{plots/Extrap_Zee.eps}
		\end{center}
	\cright{.5}
		\begin{center}
		$\zprime \rightarrow \mu\mu$
		\includegraphics[width=\linewidth]{plots/Extrap_Zmm.eps}
		\end{center}
	\cend
\end{frame}


\begin{frame}
	\frametitle{Method For Setting $\mu$ Range}
	
	\cleft{.5}
		\begin{center}
		$\wprime\, \rightarrow e\nu$
		\includegraphics[width=\linewidth]{plots/Extrap_Wenu.eps}
		\end{center}
	\cright{.5}
		\begin{center}
		$\wprime\, \rightarrow \mu\nu$
		\includegraphics[width=\linewidth]{plots/Extrap_Wmnu.eps}
		\end{center}
	\cend

\end{frame}


\begin{frame}
	\frametitle{1. CL Calculation}
	\cleft{.65}
		
		\vspace{20pt}
		
		\li{n pseudo-experiments for each scale factor for each mass point.}
		
		\li{Likelihood function is chosen - ratio of profile likelihoods :}
		\vspace{10pt}
		{\centering{
		{\footnotesize{
		\begin{equation*}
			\log{\frac{L(\mu_{test}, conditional\,MLE\,for\,test\,nuisance)}{L(\mu_{null}, conditional\,MLE\,for\,null\,nuisance)}}
		\end{equation*} }}
		}}
		\vspace{-5pt}
		
		\li{f(q|s+b) and f(q|b) are calculated for each scanned POI value}.

	\vspace{-15pt}
	\begin{center}
	{\footnotesize{
	\begin{minipage}{.5\linewidth}
		\begin{equation*}
			\begin{split}
				CL_{s+b} = \frac{ \int_{min}^{q_{0}} test }{ \int_{min}^{max} test } \\
				CL_{b} = \frac{ \int_{min}^{q_{0}} null }{ \int_{min}^{max} null } \\
		\end{split}
		\end{equation*}	
	\end{minipage}\hfill
	\begin{minipage}{.5\linewidth}
		\begin{equation*}
 			CL_{s} = \frac{CL_{s+b}}{CL_{b}}
  		\end{equation*}
	\end{minipage}
	
	}}
	\end{center}
		
		
	\cright{.35}
	\begin{center}

		\footnotesize{m = 1 TeV, $\mu$ = 0.00201463479978}
		\includegraphics[width=.9\linewidth]{plots/Limits/StatOnly_14STEP_100PE100_1000/plots_StatOnly_14STEP_100PE100_1000/peCombined_StatOnly_14STEP_100PE100_1000_1000_0.00201463479978_c3b.eps}	
		
		\footnotesize{m = 1 TeV, $\mu$ = 0.000604390439934}
		\includegraphics[width=.9\linewidth]{plots/Limits/StatOnly_14STEP_100PE100_1000/plots_StatOnly_14STEP_100PE100_1000/peCombined_StatOnly_14STEP_100PE100_1000_1000_0.000604390439934_c3b.eps}	
		
	\end{center}
	\cend
	
\end{frame}





\begin{frame}
	\frametitle{2. Finding the Value of mu}
	\vspace{20pt}
	

	\li{For each mass value, a series of signal scaling factors are tested.}
	\li{The $\mu$ value is chosen by extrapolating a fit for these points at p=0.05 (95\% CL).}	
	
	\cleft{.5}

	\vspace{10pt}
	\centering{1.5 TeV}
	\vspace{10pt}
	\includegraphics[width=.9\linewidth]{plots/Limits/StatOnly_14STEP_100PE100_1500/plots_StatOnly_14STEP_100PE100_1500/Overview_1500.eps}


	\cright{.5}
	
	\vspace{10pt}
	\centering{4 TeV}
	\vspace{10pt}
	\includegraphics[width=.9\linewidth]{plots/Limits/StatOnly_14STEP_100PE100_4000/plots_StatOnly_14STEP_100PE100_4000/Overview_4000.eps}
	


	\cend
\end{frame}


\begin{frame}
	\frametitle{Timing}

	
	\li{Plots \& logfiles are produced to monitor the time taken for each job.}
	\vspace{10pt}
	\cleft{.5}
	\vspace{15pt}
	\includegraphics[width=.9\linewidth]{plots/Limits/StatOnly_14STEP_100PE100_5000/plots_StatOnly_14STEP_100PE100_5000/TIMEreal.eps}
	
	\cright{.5}
	
	\li{Example - 5 TeV mass point, 100 PE's with no systematics.}
	\li{Plot shows point for each job (100 jobs) for each SF.}
	\vspace{10pt}
	\includegraphics[width=.9\linewidth]{plots/Limits/times5TeV.png}
	
	\li{Just ran a test for a workspace with systematics (except MC  stat. uncertainty).}
	\lii{Jobs took ~1.2 seconds each.}
	
	\cend

\end{frame}