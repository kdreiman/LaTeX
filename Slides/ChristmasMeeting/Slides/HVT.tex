\begin{frame}
	\frametitle{Model: HVT}
	
	\cleft{.5}
	
	\li{In general effective theories with an extended gauge sector, new particles can arise in multiplets of Lorentz and gauge quantum numbers.}
	\lii{Where each multiplet has a well defined phenomenology.}
	\li{We consider a \href{https://arxiv.org/abs/1402.4431}{{\color{ATLASBlue}\underline{Heavy Vector Triplet}}} $\mathcal{W}$.}
	\lii{Predicts two charged \wprime s and an uncharged \zprime\, which are degenerate.}
	\lii{Three couplings: to leptons $g_{l}$, quarks $g_{q}$ and Higgs $g_{H}$.}
%	\li{Singlet $\mathcal{B}$ could also be considered, though this is only possible in neutral channels and involves six couplings.}
	
	\cright{.5}
	\begin{center}
	\includegraphics[width=\linewidth]{plots/MPV_quantumnumbers.jpg}\\
	\scriptsize{\color{ATLASBlue}Manuel P\'erez-Victoria}
	\end{center}
	\cend

\end{frame}




\begin{frame}
	\frametitle{Model: HVT}
	
	\cleft{.4}
	
	\li{Many available channels to combine.}
	
	\li{Appropriate for $\ell\ell/\ell\nu$ \& VV combination effort.}
%	\lii{Other channels ($\tau\tau, tt, tb, jj, bb$...) could be added.}
	
	\li{Limits are set in a coupling plane.}
	
	\li{Must ensure that combined results are subject to the same assumptions.}
	\lii{Dilepton: $g_{H}=0$, \\Diboson: $g_{H}!=0$ {\color{red}!}}
	
	\cright{.6}
	\vspace{10pt}
	\begin{center}
	
	\includegraphics[width=\linewidth]{plots/MPV_HVT.jpg}\\
	\scriptsize{\color{ATLASBlue}Manuel P\'erez-Victoria}
	\end{center}	
	\cend
	
\end{frame}



\begin{frame}
	\frametitle{Model: HVT}
	\cleft{.4}
	\begin{center}
	\includegraphics[width=\linewidth]{plots/HVTExclusion.png}
	\end{center}
	
	\cright{.6}
	\li{Limits in the parameter space of the HVT model are set by the \href{https://cds.cern.ch/record/2273871/files/ATLAS-CONF-2017-055.pdf}{{\color{ATLASBlue}\underline{heavy diboson resonance search}}}.}
	
	
	\li{$g_{f} = g_{q} = g_{l} = \frac{g^{2}c_{f}}{g_{V}}, g_{H} = c_{H}g_{V}$}
	\lii{$g =$ SM $SU(2)_{L}$ gauge coupling.}
	\lii{$g_{V} $ parametrizes  interaction strength between the heavy vectors.}
	\lii{$c_{f,H} \rightarrow$ free parameters fixed in the explicit model.}
	
	\li{Range of couplings considered usually limited to $g_{f} <$ 0.8 to remain in the region with relatively narrow resonances $\bigg( \frac{\Gamma}{M_{pole}} < 5\% \bigg)$.}

	\li{We consider signals in the context of HVT A ($g_{V}=1$).}
	\cleft{.5}
	\lii{$g_{l} = g_{q} =$ -0.554}
	\cright{.5}
	\lii{$g_{H} =$ -0.56}
	\cend
%%	Signal templates for the limit setting are produced for HVT A (gV = 1)
%%43 coupling point with gl = gq = −0.554 and gH = −0.56
	
	\cend
\end{frame}


\begin{frame}
	\frametitle{Signal modelling \& Validation}
	\vspace{10pt}
	\li{Signal templates are produced by reweighting Leading-Order (LO) Drell-Yan samples (PYTHIA 8) using \href{https://twiki.cern.ch/twiki/bin/viewauth/AtlasProtected/LPXSignalReweightingTool}{{\color{ATLASBlue}\underline{LPXSignalReweightingTool}}}.}
	\li{Tool was updated to include HVT model with $g_{H}=0$ \& $g_{H}!=0$.}
	\li{Validated (\href{https://indico.cern.ch/event/632029/contributions/2555625/attachments/1444112/2224817/ZpSignalTemplates_Dan_Peter.pdf}{{\color{ATLASBlue}\underline{$\ell\ell$}}}, \href{https://indico.cern.ch/event/632029/contributions/2555626/attachments/1444674/2225285/wprime1670413.pdf}{{\color{ATLASBlue}\underline{$\ell\nu$}}}) for a wide range of couplings against Madgraph5 and PYTHIA 8 MC samples at truth-level.}
	
	\vspace{-15pt}
	\begin{center}
	
	\includegraphics[width=.7\linewidth]{plots/SignalValidation3TeV.png}
	\includegraphics[width=.25\linewidth]{plots/SignalValidation2TeVW.png}
	
	{\scriptsize {\color{ATLASBlue}P. Falke, D. Hayden, Y. Takubo \& K. Krowpman}\\
	Black dots = PYTHIA 8, blue = reweighting result.}
	\end{center}
	\vspace{-20pt}
\end{frame}











\begin{frame}
	\frametitle{Interference Effects}
	
	\cleft{.4}
	\vspace{10pt}
	
	\li{Interference has constructive or destructive effects, depending on the sign of ($g_{q}, g_{l}$).}
		
	\li{Not much dependence on $g_{H}$.}
	
	\li{Large and non-negligible impact for width of \textasciitilde 2.5\%.}
	
	\li{Not enough time to fully implement interference effects into limit setting for current analysis.}
	\lii{Instead, truncate templates to avoid regions which are most affected.}
	
	\cright{.6}
	\vspace{10pt}

	\begin{center}
	($g_{q},g_{l}$) = (-0.554, -0.554). 
	\vspace{-10pt}
	\includegraphics[width=\linewidth]{plots/InterferenceEffects.png}\\
	\vspace{-5pt}
	{\scriptsize {\color{ATLASBlue}P. Falke \& Y. Takubo}}
%	\href{https://indico.cern.ch/event/675849/contributions/2777367/attachments/1551675/2437954/Width__Cross_Section_update.pdf}{{\color{ATLASBlue}\underline{Kyle Krowpman, Daniel Hayden}}}

	\end{center}
	\cend
	

\end{frame}











