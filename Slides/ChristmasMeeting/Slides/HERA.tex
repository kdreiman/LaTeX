\begin{frame}
	\frametitle{HERA PDF}
	
	\li {In the case of HERA 2.0, two error sets are provided:}
	\lii {An asymmetric Hessian set of 28 EVs.}
	\lii {A set of additional variations:}
	\liii {10 model variations which are paired}
	\liii {An envelope of 3 maximal parametrisation variations}
	\li {These values must be processed before being added in quadrature in order to obtain the full up/down error.}
	
\end{frame}



%The positive (negative) model errors are obtained by taking the difference between each of the ten variations and the central value and then adding together all of the positive (negative) differences in quadrature. The largest positive (negative) difference between each maximal parametrisation variation and the central value is then taken as the positive (negative) parametrisation error and is added in quadrature to the model error to form the parametrisation envelope. 
% 
%  $mem=0 => central (fs=0.4,mb=4.5,mc=1.43,q20=1.9,q2min=3.5,a_s(MZ)=0.118);$
%  $mem=1 => fs=0.3;$                
% $mem=2 => fs=0.5;$
%  $mem=3 => fs=hermesfs-03;$        
% $mem=4 => fs=hermesfs-05$
%  $mem=5 => q2cut=2.5;$             
% $mem=6 => q2cut=5.;$
%  $mem=7 => mb=4.25;$               
% $mem=8 => mb=4.75;$  
%  $mem=9 => mc=1.37;$               
% $mem=10 => mc=1.49;$
%  $mem=11 => par2(Q0 1.6, mc1.43);$ 
% $mem=12 => par3 (Q0 2.2, mc1.49);$
%  $mem=13 => par4(Duv);"$